\documentclass[10pt]{article}
\usepackage{latexsym,amsmath,amssymb,amsfonts,fullpage,times}
\usepackage{url}
\usepackage[english]{babel} 
\usepackage[pdftex]{graphicx}
\usepackage{color}
\usepackage{wrapfig}

%\usepackage{amsthm}
\usepackage{url}
\setcounter{tocdepth}{3} 
\usepackage{graphicx}

\usepackage[sc]{mathpazo}
\newenvironment{itemize_packed}{
\begin{itemize}
  \setlength{\itemsep}{1pt}
  \setlength{\parskip}{0pt}
  \setlength{\parsep}{0pt}
}{\end{itemize}}

\linespread{1.1} % This can go down to 1.05 and still be within spec

\oddsidemargin 0.0in
\evensidemargin\oddsidemargin
\setlength{\textwidth}{6.5in}
\setlength{\textheight}{9.0in}
\setlength{\topmargin}{0.0in}
\setlength{\headheight}{0.0in}
\setlength{\headsep}{0.0in}

\newcommand{\todo}[1]{\textcolor{red}{#1}}
\newcommand{\TODO}[1]{\todo{#1}}

\begin{document}
\thispagestyle{empty}
%\maketitle
\begin{center}
{\large\bf Collaborative Proposal:  SI2-SSI: Title Goes Here}

\vspace{1ex}
PI: 
\end{center}

%% The Project Description should provide a clear statement of the
%% work to be undertaken and must include: objectives for the period
%% of the proposed work and expected significance; relation to longer-term
%% goals of the PI's project; and relation to the present state of
%% knowledge in the field, to work in progress by the PI under other
%% support and to work in progress elsewhere.

%%% Introduction/Overview of the Proposal

Provenance is metadata that describes the history of a digital object:
where it
came from, how it came to be in its present state, who or what acted upon it,
etc.  \todo{I wouldn't have "etc." in the opening sentence. -- Aaron}
Provenance is widely recognized as an essential means to document and cite
computational
It also adds value to data and experiments by providing a means to experimental
reproducibility, facilitating precise tracking of data sets and computational
artifacts, and enhancing data validation.
There exists a community devoted to defining,
formalizing, and standardizing 
provenance~\cite{buneman-archive,buneman-ww,cheney-da,cheney-wwh,opm}
as well as designing and developing systems that
capture and record
provenance~\cite{starflow,orchestra,provchallenge1,pass-usenix06,trio}.
Despite the needs of the scientific community and the activity of the
provenance community, there has been only limited scientific impact from
the advances in our understanding and management of provenance.
At the Workshop on the Theory and Practice of
Provenance held in 2011~\cite{tapp11} \todo{Aaron: Given that we are up to what, TAPP 13, and all the activity, it indeed seems odd that there has been limited scientific impact from provenance work..  Response:  I think the 
number refers to the year  I
believe TAPP 2014 is the 6th TAPP workshop},
the community agreed that facilitating and encouraging adoption of
provenance was one of the most significant challenges ahead~\cite{chapman11}.

Most existing provenance systems require users to adopt a particular tool
set in order to benefit from provenance.
For example, Starflow requires that you use the Python
language~\cite{starflow},
Trio requires that you use the Trio Database Management System~\cite{trio},
Kepler~\cite{kepler},
Vistrails~\cite{vistrails-sigmod06,vistrails-ssdbm08},
Pasoa~\cite{pasoa} and
myriad other workflow engines all offer provenance support if you use
a particular workflow engine,
and PASS requires that you run a modified operating
system~\cite{pass-usenix06}.
This approach presents two problems to the adoption of provenance technology
by scientists.  First, it requires scientists to learn a new technology, since provenance
collection is generally not available in the environments with which they are most familiar.
Second, provenance collected from one technology cannot be easily integrated
with provenance collected from a different technology even if they both collect provenance 
and are used in combination by the scientist to accomplish a task.
Our goal is to bridge the gap between the ``use-my-system'' provenance
solutions that exist and the reality of scientists who work in multiple
environments, use whatever languages and tools they choose, develop
their own computational tools, use a variety of extant data sources, and who
would rather spend their time doing their scientific work than in learning a new
technology.

The Harvard group, led by Seltzer, has many years of experience 
developing the provenance aware
storage system (PASS) and integrating it with a number of other provenance
solutions.
The team from Mount Holyoke College (MHC), led by Lerner,
has been working on two tools to support data provenance.
The first is a library used to collect provenance from R~\cite{R}, a language that is widely used 
by scientists \todo{in disciplines ranging from... to...} to do data analysis.
The second is a visualization tool that has been designed to be language-independent and 
has so far been used with provenance collected from R and from Little-JIL~\cite{LitteJIL}.
The MHC team has been working with the team from 
Harvard Forest (HF), led by Boose and Ellison, for over ten years,
on studying the provenance needs of environmental scientists, working closely on
the development and evaluation of the R library in the past year.
This proposal combines these related efforts to bring a powerful collection of
integrated provenance tools and capabilities into the hands of domain scientists
in a manner that makes them easy to adopt and use.
The end result will include a library for use
by any software developer to collect and use provenance,
a demonstration of that library in both the R and Little-JIL \todo{Barb: How important 
is Little-JIL to  this proposal?}
languages, and multiple science applications \TODO{MIS: Can we be more
specific about the applications} using these
libraries.
The different teams serve as platforms and testbeds for one another:
The MHC team is a testbed and critical audience for the Harvard libraries,
while the Harvard Forest team is a testbed and critical audience for the
R and Little-JIL tools.
The final test is the value that added provenance capabilities bring
to the scientific results produced by the Harvard Forest team.

Accomplishing the goals of this proposal requires fundamental research
in the following areas.

\textbf{Reconciling provenance at multiple semantic granularities.}
Different layers of a software stack manipulate different kinds of
objects.  Databases manipulate tuples using relational algebraic
operations; operating systems manipulate files and processes;
languages operate on variables and operators; workflow engines
manipulate objects and messages; and biologists experiment at scales
ranging from base pairs and genes to planetary ecosystems.  We will
develop models and mechanisms that relate objects at different
layers to facilitate capturing and querying their provenance.

\textbf{Capturing detailed provenance at the level of a programming
language.}  Most provenance is captured at a coarse grain, such as
files or components in a workflow.  Understanding how a file came
into existence or what a workflow component does requires more
detail.  We will develop an approach that collects sufficient
provenance data to provide this detail while scaling well as the
volume of data increases.

\textbf{Defining precisely aspects of provenance required
for different use cases.}
Provenance can be used in myriad ways, and a system must capture
different information depending on the intended use, which might
vary during the project's lifecycle.  We will formalize the
relationship between what must be collected and the uses to which
the provenance will be applied.

\textbf{Making provenance accessible to scientists.}   Building a
successful provenance system is more than a technology problem.  It
depends on collecting the right data and presenting it to the
scientist in a way that is meaningful and useful.  For example, the
most common provenance query we encounter in talking with scientists
is the ancestry query, "Where did this item come from?" This question
is ill-posed: from what point in time should we return an answer?
Understanding the context of this question and the data being queried
will aid in determining the \emph{right} provenance to return.  We
will work closely throughout this project to ensure that the software
we develop addresses the real needs of real scientists.

To address these research questions, we will focus our efforts on the following technologies:

\textbf{API Design: }The Harvard team developed the Disclosed Provenance
API as part of its layered provenance architecture.  This provides a good
starting point, but is currently insufficient to handle the interaction
of related objects at different layers and the multi-granularity provenance
required for this proposal.

\textbf{RDataTracker library:}  The R library developed by the MHC
and HF teams uses a combination of instrumentation by the scientist,
introspection, and source code analysis.  We plan to continue this
work, simplifying the scientist's instrumentation task, improving
the precision of the provenance collected, in a manner consistent
with important aspects of R, such as its support for lazy evaluation
and using the API developed by the Harvard team.

\textbf{Scientific applications using provenance:}  The HF team
will focus on the provenance use cases, using RDataTracker and other
tools integrated with the Harvard library to help address the
question of the value of provenance to the working scientists.

%% Intellectual Merit
\textbf{Intellectual Merit: }
\todo{This para seems off target to me. In my experience, the Intellectual Merit should focus on the core disciplinary (here CS) questions of interest at a fairly high (theory) level. "Synthesizing research" is not quite the right flavor. Resolving provenance relationships, semantics, and granularity seem more on target. So they should probably be the emphasis here. -- Aaron}
The intellectual merit lies in bridging the gap between \emph{provenance
capture} and \emph{use}.
That is, while many existing systems capture provenance, there are few
documented cases of provenance use, suggesting that fundamental research
is required to bridge the gap between \emph{data} and \emph{information};
that is the major thrust of this work.
Resolving the provenance relationships between different semantic levels
and object granularities is the cornerstone of this work.
As existing systems focus exclusively on a single semantic level or
unit of granularity, this problem remains open today and is critical
to enable the use of provenance in heterogeneous environments.

%% Broader Impact
%% Provide a compelling discussion of the software's potential use by
%% broader communities, preferably via use cases developed in concert
%% with relevant domain scientists

\textbf{Broader Impact: }
The potential impact of this work is enormous.  Although we focus
on applications in environmental science
\todo{I can provide
data/platforms from 'omics (proteomics/transcriptomics) as well. I
would think that breadth would be appreciated by reviewers. --
Aaron; Absolutely! (MIS)},
the tools that we develop and insight we gain are general-purpose.
By making the tools widely available, we expect that
they can and will be adopted by scientists in other
disciplines.  Other colleagues at Harvard in physics and imaging
have also expressed interest in such capabilities.  Seltzer's group
at Harvard is currently engaged in a data mining project for which
provenance is being recorded; we anticipate converting
to the tools developed in this project.
The R library has already
been described at meetings and is gaining interest from users outside
of Harvard Forest and we expect to present workshops at ecology
meetings to promote its use.

This work also supports NSF's mission of broadening participation
in computer science as Mount Holyoke is a women's college and
therefore women undergraduates will be involved in this research.
We also expect to fund students to work at Harvard Forest in the
summer, where there has been a successful REU program for 25 years.

%% Seltzer's group has an established track record reruiting and retaining
%% women at both the undergraduate and graduate level, and we anticipate
%% that the kinds of questions posed by the system's biologists will
%% attract women to this project. An unprecedented 42\% of Harvard's newly
%% declared computer scientists are women, and we will recruit heavily from
%% this pool.


%% 
\section{Provenance and Science: A Critical Need}
\label{sec:need}

As computational scientists, we need provenance
every time we analyze data. 
A profound and tragic example of this need comes from the now-debunked study by
Potti et al~\cite{pmid17057710} (retraction~\cite{pmid21217686}),
reporting a
gene expression signature that could guide the choice of chemotherapy
for cancer patients.  After this method was put into clinical
practice, serious flaws were found in the analysis, and the test
proved to be worthless. The Institute of Medicine (IOM) initiated
an investigation to determine what went wrong, and what should be
done to avoid such serious errors in the future.
A member of the IOM investigating committee said,
"many of the problems could have
been avoided, or at least better identified, if there was a clear
record of where and when different versions of the primary and
derived research data existed".  In other words, had a provenance
system been in place, it may have prevented this tragedy.

\textcolor{magenta}{Another example was discussed by Ellison et al
(2006).  The relationship between increasing atmospheric concentration of
carbon dioxide ($CO_{2}$) and its relationship to rising global
temperatures is well understood and not especially controversial
<\textbf{cite IPCC 2014 report here}>. Forests in the northern
hemisphere remove large amounts of $CO_{2}$ from the atmosphere when
deiduous trees leaf out in the spring and photosynthesize throughout
the summer. In the winter, when leaves are shed and branches are bare,
cellular respiration predominates, and $CO_{2}$ is released back to the
atmosphere. The annual net ecosystem exchange (NEE) is the cumulative
amount of $CO_{2}$ taken up or released by forests. Precise and accurate
measurements of NEE are critical not only so researchers can
understand whether forests can buffer anthropogenic activities and
ameliorate  change but also are used to set prices in regional,
national, and international carbon markets and to define the value of
forests for carbon offsets <\textbf{this needs a reference}>.
}

\textcolor{magenta}{NEE is measured using eddy covariance methods 
<\textbf{cite Barford et al. 2001}>; measurements are taken at frequencies of
5-20Hz, averaged over 15-60 minute intervals, and integrated over an
entire year.  The accuracy of NEE estimates depend on atmospheric
conditions, however, and as much as 75\% of eddy-covariance data
streams may be unreliable or missing; these data gaps are filled using
a variety of investigator-dependent algorithms. Precise provencance is
needed to identify data gaps and modeled values; to allow for
propagation of uncertainty in annual integration and estimation; and
to allow for models to be re-run as additional data accumulate,
gap-filling algorithms change and improve, or intersite comparisons
are attempted using common methods <\textbf{some citations useful}>}


%%\todo{Another example might be the one described by Aaron in the Ecology(?) paper
%%about how different people arrive at different interpretations of the same data because they do 
%%different data analyses. MIS: Yes, let's include that as a second example so we
%%have two strong examples of the critical need and can then lead seamlessly into the
%%next section.}

\todo{Could keep the following or use similar examples.  Points we may want 
to make are the following:}

\begin{itemize}
\item \todo{Having just the source code may not be enough.  Random numbers, user input, 
  data downloaded from the Internet, etc. must be included.}
\item \todo{Source code describes what might happen.  A detailed provenance trace describes
  what actually happened.}
\item \todo{A provenance trace may be easier to understand than source code.}
\end{itemize}

The previous examples illustrate today's reality in computational science:
vital provenance data is rarely even collected, let alone used to validate,
authenticate, and reproduce scientific results.
%% At this point, we don't
%% have a solid list of requirements - producing a good requirements
%% definition is a key goal of the project -  but we do have a general
%% sense of what we need.
Most fundamentally, this gap identifes
the need for a provenance system to track what computational
scientists do: the programs run, their parameters, inputs and outputs.
And we need to be able to do so in a domain scientist's \emph{native}
environment -- forcing the use of particular workflow engines or systems
makes the barrier to entry too high.

Beyond these basic capabilities, it would be helpful to be able to
annotate data and workflows, textually, verbally (with audio), and
visually (with video) to explain why experiments were conducted
the way they were.
It would also be beneficial to capture provenance
when exploring the data,
moving dynamically through the space of possible analyses following
interesting leads as they emerge. Scientists frequently run predefined
workflows repeatedly - hundreds or thousands of times - with different
parameter settings to systematically explore the analysis space.
Similarly, many analytical tools include a stochastic element, so it
is useful to run a workflow repeatedly with the same parameters but different
random number sequences, to characterize the
distribution of results.
Finally, scientists need a way to track changes in third party
or external data and resources that are incorporated in analyses.
It is vital to track whether changes in these inputs affect results.
% Sometimes, we're able to swap components in and out
% of a predefined workflow, e.g., to try out different sequence alignment
% algorithms when working with RNA sequence data. And, since we develop
% much of the software that we use, we also modify components and
% rerun analyses on the fly.

\todo{Research questions:}

\todo{The most fundamental research question we plan to address is how to bring provenance 
to the scientists, making it easily available and accessible to the scientists.  }

\begin{itemize}
\item \todo{What information should be collected?}
\item \todo{What should the scientist's role be in provenance collection?}
\item \todo{What is the best way to scale up?}
\item \todo{How will scientists use provenance once they have it?}
\end{itemize}

\todo{Potential uses of provenance:}
\begin{itemize}
\item \todo{Short-Term (months)}
\begin{itemize}
\item \todo{Develop and troubleshoot scripts}
\item \todo{Examine derivation and use of particular data values}
\end{itemize}

\item \todo{Mid-Term (years)}
\begin{itemize}
\item \todo{Understand original analysis}
\item \todo{Reproduce original analysis}
\item \todo{Examine derivation and use of particular data values}
\item \todo{Better use data in subsequent analyses}
\item \todo{Better use analysis in other applications}
\end{itemize}

\item \todo{Long-Term (decades)}
\begin{itemize}
\item \todo{Same as mid-term, but replication may no longer be possible}
\item \todo{Source code and data provenance will be key to understanding original analysis}
\end{itemize}
\end{itemize}

In designing the project, we chose an initial set of applications
and use cases that we believe can be solved and whose solutions
will help us and others in the field move on to more challenging cases.
The initial applications we've selected are: 
\todo{???}.

\section{Provenance in Environmental Science}

%% This is a section I'll need someone else to write.   I think it comes
%% BEFORE I go into detail on the research we want to do, because I'm hoping
%% it will motivate that research and I can tweak what I have to speak to
%% the uses directly
%% The vision I have for it is something like this:


\subsection{Use Case 1 - Modeling hydrology \todo{Emery/Barbara to write this one}}


\subsection{Use Case 2 - Modeling forest carbon flux and estimating annual net ecosystem exchange}

\subsection{Use Case 3 - Using `omics to forecast evolutionary responses to climate change}

\textcolor{magenta}{As the Earth's climate changes globally, habitats
that are home to distinct species are changing too. The organisms that
live and thrive in a particular location can respond to locally
changing climates in one of four ways: they can move (migrate), stay
put and acclimate (persist), go extinct, or evolve into new
species. <\textbf{insert ``climate cascade'' figure \ref{fig:climate-cascade} here}>. Most
research in this area has focused on how or when organisms might track
climate change by migrating to new habitats (e.g., north or south, or
up mountains, to remain in thermal equilibrium) <\textbf{general
  review of SDMs}>, but new research suggests that genetic variability
within a species may allow individual lineages to acclimate and
persist in place. A research team co-led by co-PI Ellison and
supported by NSF Dimensions of Biodiversity award 1136646 is using
genomic, proteomic, and transcriptomic data collected from ants
throughout eastern North America to study how evolution of heat-shock
proteins and temperature-induced changes in protein expression and
protein conformation affects performance of individual species in
different climates.}

\begin{figure}
\includegraphics[width=\textwidth]{figs/climate-cascade-figure.jpg}
\label{fig:climate-cascade}
\caption{\small{Caption goes here.}}
\end{figure}

 \textcolor{magenta}{Joint analyses of genomic, proteomic, and
transcriptomic data are uncommon and present novel bioinformatic
challenges that would benefit from provenance. At a minimum, each of
these sets of `omic data (Gigabytes-to-Terabytes) must be assembled,
checked, and annotated using standard methods, but different standards
apply to different `omes. Heat-shock proteins are conserved across
animals and plants <\textbf{citation}>, but ant `omes are much more
poorly studied than those of (e.g.) people, mice, or rats, and the
data assembly pipeline requires much more ``hands-on'' work and
(re)checking than a ``standard'' BLAST search for, say, a new human
cancer gene. And completing the genome, proteome, or transcriptome of
even a single species of ant is just the first step. Modeling protein
conformations and their responses to different climate-change
scenarios requires extensive custom coding (and debugging) to ensure
reliable results. At each stage, collection of detailed provenance
would ensure reproducibility of analysis, reliability of results, and
information useful to track uncertainty through the process and to
associate levels of uncertainty with the different potential outcomes
for individual species in a changing climate.}

\section{Research Plan}

\todo{Need to subdivide into 2 sections.  One R-specific, 
one for the interoperability toolkit (currently here)}

The team at Harvard is responsible for the computational research necessary
to produce the software artifacts to be incorporated by ISB researchers
as described above.
While the Harvard team has extensive experience producing provenance
solutions in tools, languages, and operating systems, their projects to
date all require that users adopt a particular tool set or environment.
We refer to that model of provenance provision as ``bringing the users to
the tools''; our goal is to bring tools to the users by
producing a provenance toolkit, easily integrated into any desired platform.
As described in the introduction, designing such a toolkit requires
fundamental research in four areas: reconciling provenance at multiple
semantic granularities, identifying what provenance is required for
what capabilities, returning appropriate provenance in response to
queries, and API design.

The first two subsections that follow present the foundational motivation,
experience, and model for the research.
We next present a brief architectural description of the provenance library.
The subsequent four sections then outline our research approach to
each of our four challenges.

\subsection{Layering in PASS}
\label{sec:pass}

The layered PASS architecture addresses three specific problems in
provenance integration: data/provenance consistency, stacking,
and cycles.

%% Data provenance consistency
Data/provenance consistency allows a system to make concrete statements
about the semantic consistency provenance provides.
\emph{Complete} provenance means that the provenance present in a
system describes all the transformations applied to the data it describes.
\emph{Correct} provenance means that all the transformations present in
the provenance were actually applied to the data.
PASS ensures complete provenance via its \emph{Disclosed Provenance API}
(DPAPI), which transmits data and provenance atomically, and its use of
write-ahead provenancing (WAP, akin to write-ahead logging), which guarantees
that even after a crash, a description of every transformation to a piece of
data is present in the persistent provenance store.
PASS ensures correctness using hashing to identify cases where provenance
was written to a persistent store, but the data described by that provenance
has not yet made it to the persistent store and may have been lost in a
failure.
Using a transaction-like mechanism during recovery, PASS provides
both completeness and correctness.
We will retain WAP and an integrated data/provenance API in this work.

The DPAPI does not, however, capture the rich set of relationships
among objects at different semantic levels, such as containment,
encapsulation, specialization, and generalization.
Therefore, one of the challenges to be addressed is to identify the
requisite relationships and then develop
a more general and flexible API to express those relationships,
retaining the atomic flow of data and provenance and the WAP capabilities.

%% Stacking

In building provenance-aware applications and components, we quickly
realized that some components are both consumers and producers of
data and therefore must consume, transmit, and produce provenance
as well.
Thus, the provenance architecture must facilitate \emph{stacking} such that
a component can accept provenance from another component, augment that
provenance, and then transmit it to yet another component.
For example, a data pipeline that ingests a file containing both
organism and gene data,
transforming it into a per-organism and per-gene databases,
must accept provenance about the file, retain that provenance,  and
produce provenance about the various tables and tuples it creates,
linking the different objects correctly.
%% The integrated system must properly track provenance from the database
%% and from end-products derived from the database, back to the original
%% file and its ancestry.
Thus, like the DPAPI, the API we design here must stacking provenance:
receiving, augmenting, and then transmitting provenance between
software tools and layers.
Doing so in a way that preserves key properties of provenance and
facilitating semantically meaningful queries across software
components remains an open problem.

%% Object Identity

%% Nat -- I'm guessing you folks might run into this when you have data
%% whose ancestry might overlap with data you directly download?
%% I'm kind of winging this here -- does this happen with the data you
%% process?  Might it make sense to include a simple figure with some
%% actual cases where this does/might happen?
%% For example, the team from ISB regularly downloads dozens of publicly
%% available data sets.
%% Some of those data sets are themselves the result of data synthesis from
%% other data sets.
%% The ISB team then uses these data sets to create new data sets.
%% Today, many of these data sets do not carry their full ancestry with them,
%% so there is no way to know if the same data set acts as multiple inputs
%% to some output dataset.
%% Retaining this knowledge would be of immediate benefit.

%% Cycles

Making connections between objects becomes more challenging in distributed
environments where objects bear different names.
The Second Provenance Challenge~\cite{challenge2} identified object
naming and name resolution as one of the most challenging aspects of
provenance interoperability.
Resolving such naming challenges is critical for avoiding provenance cycles.
Provenance must form a directed acyclic graph.
Since provenance represents ancestry, it is crucial that the graph is
acyclic, or else it would suggest that multiple items were related both
through ancestry and descendency relationships,which is problematic.
However, systems that infer provenance from observed events (e.g., PASS)
will produce cyclic graphs unless care is taken to avoid them.
For example, a process tracking file-level provenance that reads and writes
from/to the same file creates cyclic dependencies.
Versioning objects is the obvious way to deal with cycles, and PASS has
evaluated multiple versioning approaches~\cite{causalversioning-fast09},
but none of them can adequately address the challenge that arises in
avoiding cycles that can occur between objects created at different
layers of software and different granularities.

Addressing this problem relies on developing solutions to the following
challenges:
we need to identify the range of relationships to express,
design a means of expressing those relationships, and
resolve names between the various layers.
Furthermore, we need to develop a formal model of provenance that will
let us make precise statements about the semantics of provenance and
therefore define how provenance from different levels can and must interact.
Current models~\cite{opm,semirings,buneman-ww,cheney-wwh} are
insufficient to capture the broad range
of relationsihps that we have identified to date.

\subsection{A Graphical Model of Provenance}
\label{sec:graph}

In the database world, provenance results only from the execution of
queries.  In this constrained world, provenance can be expressed in
terms of two operators that form a semi-ring~\cite{semirings}.
In the wild, provenance results from any computation and cannot
be so cleanly expressed.
Thus, we model provenance as a graph.

A provenance graph is a labeled graph whose nodes represent both
objects and agents and whose edges represent relationships among
those nodes.
We will begin with a set of edges whose meaning is precisely defined
and then extend the model to include arbitrary edges.

\subsubsection{The Basic Model}

\todo{This is different than how we talk about provenance graphs.  We have
several types of nodes that fall into 2 broad categories:  operations and data.  
We also have 2 edge types: control flow and data flow.  Our types could easily
be attributes in your model, but I'm not sure what originator corresponds to in
our view of provenance.  Perhaps originator corresponds to an operation node?  
On the other hand, you propose a much richer set of edge types than we have.
How would a richer set like this affect R provenance or improve its usefulness?}

%% Describe provenance graphs
The model derives from our work integrating provenance
systems, developing formal methods of representing security and privacy
properties, and querying provenance graphs, but we anticipate
extending the model as necessary.
A provenance graph consists of vertices, attributes, labels and edges.
Edges are directed and bear a single label;
thus every edge has a source vertex, a label and a destination vertex.

Provenance Graph $G_P := (V_P, A_P, L_P, E_P)$
\begin{itemize_packed}
\item Vertices $V_P$
\item Attributes $A_P$
\item Labels $L_P$
\item Edges $E_P := \{src:V_P, lbl:L_P, dst:V_P\}$
\end{itemize_packed}

A vertex has two or more attributes associated with it, where an
attribute is a name:value pair.
The two required attributes are \emph{id} and \emph{originator},
where \emph{id} is an object identifier and \emph{originator} uniquely
identifies the agent responsible for creation of the vertex.
Together, the \emph{id} and \emph{originator} form a unique key.

The basic model specifices a required set of labels and attributes that have
semantic meaning with respect to graph construction and query
semantics. The extended model makes it possible for any provenance-aware
agent to specify its own additional labels and attributes, so long as it
respects the semantics of the basic model.

The \emph{attributes} defined in the basic model are:

\begin{itemize_packed}
\item{id: vertex identifier, unique within an originator}
\item{originator: agent responsible for creation of a graph element}
\item{name: a string used to describe the graph element}
\item{time: the creation time of a graph element}
\end{itemize_packed}

The \emph{labels} defined in the basic model are:

\begin{itemize_packed}
\item{input: the destination vertex is an input to the source vertex; in
other words, the source depends upon the destination}
\item{version: the destination vertex is a previous version of the source
vertex}
\item{contained: the source is contained within the destination; for
example the destination might represent an entire file, while the source
represents a piece of that file}
\item{instantiates: the source is an instance of the object represented by
the destination}
\item{maps: the source is an alternate representation for the destination
object}
\item{controlled\_by: the source is controlled by the destination}
\end{itemize_packed}

Figure~\ref{fig:model-example} shows examples of several different
edge labels.

\begin{wrapfigure}{R}{0.65\textwidth}
\vspace{-10pt}
\includegraphics[width=0.65\textwidth]{figs/model-examples.pdf}
\label{fig:model-example}
\vspace{-20pt}
\caption{\footnotesize{Edge type examples. Figure A depicts a data ingest process 
that reads an input data file and then spawns a database (MySQL) and
inserts the data into relational tables in the database.
Figure B depicts a process that is the instantiation of a workflow template.
The actual file produced by the execution of the process maps to the
one described by the workflow template.}}
\vspace{-10pt}
\end{wrapfigure}

%% \subsubsection{The Extended Model}
%% While the basic model captures the relationships that we find common
%% across different software agents and necessary to relate the objects
%% managed by different agents, it is insufficient to capture all semantics
%% needed by every agent.
%% The extended model allows agents to add their own attributes and labels.

\subsubsection{The Query Model}

In our model, any query on a directed acyclic graph constitutes a
provenance query; however there are several queries that we find
common across domains.
We have developed a path-oriented query language (PQL)~\cite{pql-manual} that is
sufficiently powerful to express such queries.
The basic query model is that of traversal across edges, with optional
matching on attributes.
Rather than going into the details of our specific query language here
(the interested reader can refer to our reference guide~\cite{pql-manual}),
we instead outline those queries we have found most common and demonstrate
how they are, in fact, straightforward graphical queries.
%% We then go on to discuss some of the challenges present in reponding to
%% such queries and our approach for addressing those challenges.

\textbf{The Ancestry Query: }
By far, the most common query we have encountered is the ancestry
query, "From where did this object come?" This query is a transitive
closure of the graph from a given node through all ``ancestor nodes.''
In the basic model, for ancestry queries, we consider input edges,
version edges, and controlled\_by edges, however in any extended
model a user or agent may specify additional edge labels (either from
the basic model or from the extended model) as representing ancestry.
The proper query result is still a transitive closure, just one along
more edges.

%% Should I go into more detail here on the prov rank work?

\textbf{Friend Queries: }
After shadowing scientific users in different fields, we found that
users frequently wanted to identify objects (in our case, files) that
had undergone processing similar to some other object.
We call such queries ``friend'' queries.
The specification of a friend query typically begins with an object in
the provenance tree and its ``immediately-relevant path.'' 
The immediately-relevant path may be specified manually or determined using
the techniques described below in Section~\ref{sec:query}.
Although most users wanted a simple path for such friend queries, we
could also begin with any designated subtree.
The chosen path is then abstracted to allow for different inputs undergoing
identical processing.
The query requests all paths matching the abstract path.
In the data ingest use case described above, the query that asks for the
friends of a particular data product should return all the data products
produced by the same ingest process.
% Figure~\ref{fig:friend-query} provides an example of a friend query.
% 

\textbf{Repeated Trial Queries: }
Repeated trials are similar to friends.  Once again, we begin with a specific
object and its immediately relevant ancestry path.
This time, we want all such paths beginning with the same input, experiencing
``similar'' transformations, where the transformations differ only in
attributes or a computational module.
For example, the ISB team frequently repeats the same analysis experimenting
with alternate analysis modules.
Given the output of such an analysis, a repeated trial query should return
all the data outputs of the analysis, including those that used different
parameters or different analysis modules.
% Figure~\ref{fig:trial-query} depicts this repeated trial query.

% \begin{figure}
% \begin{center}
% \label{fig:friend-query}
% \caption{Friend Queries: We are looking for all objects that have undergone
% processing similar to that producing OUT.
% We see N such instances, OUT-x, OUT-y ...}
% \end{center}
% \end{figure}
% 
% \begin{figure}
% \begin{center}
% \label{fig:trial-query}
% \caption{Repeated Trial Queries: We are looking for repeated
% trials of the output OUT, that differ only in the ANALYSIS MODULE.}
% \end{center}
% \end{figure}

\subsection{Library Architecture}

The provenance library consists of a set of modules that collect provenance,
construct the graphical representation as described above, and persist the
provenance to a backend database.
As our goal is to support a variety of programming languages and database
backends, the library includes a collection of language bindings and
database adapters.
Figure~\ref{fig:architecture} shows the architecture and components of the
Core Provenance Library.

\begin{wrapfigure}{R}{0.5\textwidth}
\vspace{-30pt}
\includegraphics[width=0.5\textwidth]{figs/architecture.pdf}
\label{fig:architecture}
\vspace{-20pt}
\caption{\small{Core Provenance Library Architecture.  The databases and language}
adapters shown in bold are ones that are needed for our applications.}
\vspace{-10pt}
\end{wrapfigure}

\subsection{Provenance Reconciliation} 
\label{sec:reconcile}

Provenance reconciliation is the process of integrating provenance contributed
by different software agents.
Most existing provenance systems operate at a single level of abstraction:
the system call layer, a workflow specification, or the
high-level constructs of a particular application.
The provenance collectable in each of these layers is
different, and all of it can be important.
%% Might be nice to cast this in concrete biological terms -- I'll try
%% For example, a data set, stored in a single file,
%% might include data for multiple organisms and multiple genes.
%% When a researcher downloads the entire file, the download program or system
%% has only one way to track provenance: at the unit of a file.
%% However, when a scientist later analyzes a particular organism,
%% s/he needs to track provenance at the level of the organism.
%% Similarly, analyses on genes appearing in multiple organisms require
%% yet another semantic interpretation.
The research challenge lies in expressing the relationship among these
different entities correctly and in such a way to facilitate queries
both within each semantic level as well as across levels.
Most current systems fail to account for the different levels of
abstraction at which users need to reason about their data and
processes, therefore these systems cannot integrate data provenance across
layers and cannot answer questions that require such an integrated view.

There are two fundamental approaches to provenance reconciliation.
One approach, where every software system interacts directly with every
other software system in the manner specified by each system, produces
an $n^2$ problem.
The other approach requires that the software artifacts agree how to
express provenance relationships and use a common API.
We are adopting the second approach, because it is significantly more
scalable and requiring the use of a common API allows applications
to express the relationship between objects they manage and those managed
by other software artifacts.

One of the key challenges in managing this interaction is properly accounting
for versions.
Although versioning has been fundamental to PASS since its inception, it
is only more recently that others have recognized the criticality of
versioning~\cite{lyle,fonseca,zhou}.
Versioning becomes increasingly complicated when different software layers
are managing related items and independently creating versions.
For example, consider the data ingest scenario.
The objects that are obtained from external sources are frequently flat,
structured files.
Early in the ingest process we transform those flat files into database
tables.
The modules that manipulate the database will collect provenance on
tables and will need to create versions of those tables and track
relationships among specific versions.
At the same time, we may download new versions of the source files.
Should the modules manipulating database tables need to refer back to
the original files, it is critical that we do so using appropriate version
numbers.
While manageable with only two layers of software, as the number of layers
of software increases and the relationships among the entities being managed
becomes more complex, the problem becomes more challenging.
We will draw on our experience managing versions in PASS and layered
applications to develop algorithms that ensure proper version handling.

\subsection{Provenance in support of Use Cases}
\label{sec:use}

At the most recent Workshop on the Theory and Practice of
Provenance~\cite{tapp11}, the participants identified myriad uses
for provenance: increasing trust in an artifact, validating data
currency, recovery, repeatability and exploration, annotation
propagation, citation, specification of update semantics, debugging,
schema integration, security auditing, data synchronization, self-adjusting
computation, and documenting probabilistic data.
Given this wide-ranging list of uses, it is no wonder that different
systems choose to capture widely varying amounts of provenance.
The challenge we will address is to provide the flexibility to capture
provenance to address a wide range of use cases and to specify with
precise models matching the provenance to collect with the use case.
For example, in some cases, organizations need to track where their
data goes~\cite{tapp-ucsc}, while in other cases, organizations
are strictly prohibited from tracking such information~\cite{ebi,isb,ncbi}.
Current approaches to provenance collection are haphazard in that
there is no specification of what data must be collected to facilitate
any particular use case.

% The wide-range of uses to which provenance will be applied has
% two significant effects on any provenance solution.
% First, those developing applications that track provenance must have the
% flexibility to record precisely the information in which they are interested,
% and second, systems must develop ways to limit the information returned
% in response to a query in order to uphold the data requirements of an
% organization.

We have a three-pronged approach to this problem.
First, we will formalize the various use cases and their associated
provenance needs.
Next, we will define a security and privacy model on graphical data that
supports the needs of our users.
Last, we will develop and revise our programmatic APIs throughout the
project period, to meet the needs of our users, the use cases, and
the privacy policies.
We defer discussion of the API to Section~\ref{sec:api},
discussing the first two approaches below.

Cheney's work on provenance traces~\cite{prov-traces-draft}
provides a good starting point for this work.
Cheney' work provides a formal model that shows how provenance traces created
from the execution of nested relational calculus programs produce
the three types of provenance described in the literature: why, where, and
how provenance~\cite{cheney-wwh}.
The challenge for us is to map use cases to these three provenance types,
identify gaps, use cases for which the provenance we need is none of
why, where or how provenance, and then extend the model to incorporate
those.
We have found the model effective at tackling other problems, such as
the security issues described below, which makes us optimistic that
it can be adapted for this problem.

The Harvard team has been studying the area of provenance security and
privacy and will apply graph summarization techniques to provenance graphs
to provide query capabilities under privacy policies.
We have built a prototype system that generates provenance traces
automatically from the execution of a program written in nested
relational calculus (NRC).
We then produce graph summaries from those traces that hides sensitive
information.
By allowing queries only on the summarized graph, we are able to hide
information.

Our summarization techniques draw directly upon our graphical provenance
model, described in Section~\ref{sec:graph}.
A provenance graph describes a particular provenance trace, which corresponds
to a particular execution.
A summary graph represents some number of provenance traces; the more traces
it represents, the more information it hides.
We create summary graphs by augmenting our basic graphical model with
cardinalities on vertices, edges, attributes, and labels.
We specify cardinalities via a lower and upper bound.
Thus a vertex in a summary graph with one-to-one correspondence with a vertex
in the original provenance graph will have cardinality 1:1, and a vertex
that might maps to all vertices in the original graph might have cardinality
1:infinity.
The greater the range in cardinality, the more information we can hide, but
the query results become less precise.

Our goal in this proposal is to transform the summarization and proof 
techniques we have for provenance graphs from NRC programs into real
programs and to develop summarizations that implement the specific
policies required by our sample applications.
For example, imagine that one of Gaggle's Geese records
each time it exports a data product to a user.
Now, let's say that that component is used in a system whose privacy
policy makes it unacceptable to collect or release such information.
The original provenance graph in Figure 3a
could be summarized into Figure 3b
to obscure that information.
While Figure 3b shows that the data has in fact been input to something,
the graph reveals nothing about the identity or number of such objects.
The key technical challenges to be addressed here are: 1. developing or
selecting a language in which to express the security policies;
2. automatically producing summaries from those policies; and 3.
querying on those summaries.

\begin{wrapfigure}{R}{0.5\textwidth}
\vspace{-30pt}
\includegraphics[width=0.5\textwidth]{figs/summary-graph.pdf}
\vspace{-20pt}
\label{fig:full-graph}
\caption{\small{Summarization. The graph on the left (a) shows that the data product with ID 606613 was used as input to three users.
This information is obscured in graph (b) by replacing the three nodes with
a single node and making the cardinalities of both the node and connecting
edge range from one to infinity.}}
\end{wrapfigure}

\subsection{Responding to provenance queries}
\label{sec:query}

Provenance trees can grow quite large -- the compilation of the
\texttt{am\_utils}
suite produces a provenance graph containing hundreds of thousands of nodes,
and large-scale analyses that need to model every data point easily produce
millions of nodes.
The sheer number of nodes and the fact that the graph continues to grow as
data products are used to produce new data products
introduces a second question, "How much of the tree should a query
return?"  Consider the simple ``ancestry query'' that asks where an
object came from.
The KMO example from Section~\ref{sec:natproject} 
uses a database that has been in use for approximately five years.
Had we been tracking provenance all this time and then asked for
the provenance of an element of that database, the result could
return the provenance describing each of the twenty-six versions
of the current (third) release of the database.
Alternately, an accurate, but
probably unhelpful result might include only the last modification.
In reality, the query is under-specified and it is usually not reasonable to
require that the user know the provenance tree in sufficient detail to
specify precisely how far back they would like the result to go.
We will develop techniques to return an appropriate amount of provenance
automatically in such cases.

We propose to use properties of the graph itself to determine the 
amount of provenance a query needs.
That is, for any ancestry query, there exists a subset of the entire
ancestry tree that is of interest to a user; our goal is to determine
that sub-tree of interest.
Preliminary work suggests that we can rank nodes in a provenance tree
based upon their likelihood of appearing in any ancestry query and then
use discontinuities in that ranking to identify ``good'' places to
truncate the query result.
Our work to date suggests that if we pick our ranking metric correctly,
discontinuities in that rank represent semantically differentiated points.
For example, when we applied the ranking to queries on the provenance
tree produced by the first provenance challenge, we found that the
first discontinuity corresponded well to the graph from the
original workflow specification.
The second discontinuity produces an ancestry tree including the
workflow and the compilation of the tools used to run the workflow.
The third discontinuity includes the installation of the compilation
environment itself.
Our work to date convinces us that we have two potentially good metrics,
SubRank and ProvRank,
and that with further research, we can compute them efficiently and
construct algorithms to return ``the right amount of provenance.''

SubRank is a direct measure of the frequency of a node
within the space of lineage query.
For each provenance object, simply count the number of lineages in
which it appears.
Computing this translates directly into counting
the number of descendants of the node (since the node appears in
the ancestry query of every descendant) and adding one (since every node
is in its own ancestry).
Though simple, this metric appears to work well.

ProvRank derives from the PageRank algorithm of Internet
search~\cite{pagerank}.
While SubRank measures an object's frequency of appearing in an ancestry
result directly, PageRank computes it via theoretical simulation.
If we think of PageRank as a traversal-oriented query process
in which the traversal rule is to choose equiprobably from
among the outgoing edges of each node, then we can view ProvRank as
a traversal-oriented query process in which the traversal rule is to visit
every parent.
In this case, the ProvRank is the probability that the traversal engine
considers a node at a particular instant.
When we apply this traversal-approach to a provenance graph, we find that
it produces results similar to, but subtly different from those produced
using SubRank.

The challenges in this proposal are to evaluate SubRank and ProvRank in
more depth, develop alternative metrics that work better, and devise
robust threshholding algorithms that can use these metrics to return
``the right'' portion of an ancestry tree to a user.

A second, and perhaps even more important use of these truncated
ancestries is to ``attach provenance'' to exported data products.
As discussed in Section~\ref{sec:natproject}, the exported data products
from ISB will be accompanied by provenance, but the full provenance tree
will be too large.
Using our ranking and truncation approach, we believe we can automatically
produce ``the right'' provenance to transmit with data products.

\subsection{API Design}
\label{sec:api}

While designing an API is typically considered an engineering task, our
challenge here lies in designing an API simple and flexible enough to be
adopted in a wide range of use cases, but powerful enough to support the
range of use cases to which people want to apply provennace.
The PASS Disclosed Provenance API (DPAPI), which is documented in prior
work~\cite{pass-usenix09},
provides a good starting point, but is not scalable (object identifiers
are tied to operating system resource) and does not capture the semantically
meaningful relationships that we need to capture.

The two fundamental abstractions in the DPAPI are pnode numbers, which
uniquely identify a collection of provenance, and versions.
Pnode numbers are persistent object identifiers.
Unlike many provenance systems, the DPAPI assumes that objects are
mutable and uses versioning to distinguish these mutations.
We will retain these two core concepts, but as a pnode number/version pair
uniquely identifies an object in the DPAPI, our new API will incorporate
the notion of an originator as described in Section~\ref{sec:graph}.
The originator identifies the source of a piece of provenance, which we call
a \emph{provenance record}.
In essence, the originator is provenance of a provenance record.
To avoid the potential for infinite recursion when recording provenance of
provenance, we simply trust that the originator is accurately
recorded; making concrete guarantees of that accuracy is beyond the scope
of this work.

The DPAPI always transmits data with provenance.
When data is read, the provider of the data must provide its pnode number
and version; when data is written, the writer transmits a
\emph{provenance bundle}, a colletion of provenance records describing
the pnodes on which the write depends.
However, the DPAPI does not attach any inherent meaning to the relationships
between the different pnodes in a provenance bundle.
The API that we develop in this work will use the basic model as
described earlier to ascribe specific semantic relationships between
different pnodes.
The library will then leverage those relationships to accurately
resolve the versioning challenges described in Section~\ref{sec:reconcile}
and the inter-layer cycle challenges that arise in a layered system.

%% Only six calls comprise the DPAPI; while we need to change the
%% details of the behavior and parameters of those calls to address
%% the challenges raised here, we do not yet anticipate having to add new
%% calls.
%% A small number of calls in the API is critical to facilitating adoption;
%% we believe that programmers can synthesize six function calls and learn to
%% use them almost effortlessly, while APIs significantly larger will be
%% problematic.
%% If, at the end of the project, our provenance collection API remains under
%% ten calls, we will declare success.

\section{Engineering Management and Sustainability Plan}
%% Provide an explicit description of the engineering process to
%% be used for the design, development, and release of the software,
%% its deployments and associated outreach to the end user community,
%% its interoperability with widely used tools by the community, and
%% an evaluation plan that involves end users.

%% Describe the extent to which issues of sustainability, manageability,
%% usability and composability/interoperability will be addressed and
%% integrated into the proposed software system.

%% Describe a sustainability plan for the developed software beyond
%% the lifetime of the award. Identify the open source license to be used.

By virtue of the geographical separation of ISB and Harvard, the project
will be conducted by a distributed workgroup.
%% Add any other team expertise here
Seltzer brings 15 years experience with such distributed workgroups
(Berkeley DB has always been and continues to be developed by a global
workforce).
Our plan is to use internal processes and tools that can be easily
incorporated into existing software packages.

We present the details of both our software and team management in the
supplementary \emph{Management Plan} document and focus the discussion
here on software sustainability.

\subsection{Sustainability Plan}

We are planning for a two-pronged approach to software sustainability:
first, as a standalone entity distributed by Harvard, and
second, through the existing applications developer communities that
exist for Gaggle.
%% And other tools?

Seltzer's group has a longstanding track record at producing and
maintaining software.
Berkeley DB, originally distributed by the University of California
at Berkeley, traveled with Seltzer to Harvard, where db-1.86 was
released, and was then commercialized by Seltzer and Keith Bostic as
Sleepycat Software, which is now owned and maintained by Oracle
Corporation.
In subsequent work, her group developed an instructional operating
system, OS161~\cite{os161}, that has been in active use in tens of
Universities around the world.
Through gift funding, Seltzer's group has maintained and extended
the software for the past ten years.
More recently, her group has continued to produce releases of the
PASS kernel, both as a customized Linux kernel and as a virtual
machine installation.
By continuing to use artifacts that the group produces in educational
and research activities, the software continues to be maintained.

Supporting Seltzer's efforts at releasing software, Harvard's School of
Engineering and Applied Sciences has recently announced a new program
in Applied Computational Science~\cite{iacs}.
Initially an educational program, the curriculum includes a significant
software engineering component, and the long-range plans for the
program include personnel for software support.
Seltzer served on the initial  planning team for the program and
continues to serve on its advisory board.
Our hope is that by the end of the award plan for this grant, the
library developed can become part of the Applied Computational
Science software portfolio.

The ISB team also has a longstanding record at maintaining software
for the long term.
Gaggle has been in use for over five years and has an active user and
developer community.
We will leverage this active community and the team at ISB to ensure
that provenance support becomes a standard Gaggle feature.
Additionally, we will nurture and encourage the existing community of
developers that produce Gaggle-enabled tools.

ISB also has a track record in the longevity of its data products.
Goodman's group has public databases they have maintained for over
a decade, predating Goodman's arrival at ISB.
Once we have built provenance support into those data products,
the ongoing sustenance of the data products themselves will sustain
the provenance capabilities as well.
It is precisely this Institute culture of software and data release and
sustenance that makes ISB the ideal scientific collaborator for this work.

\section{Project Plan}

%% Include a project plan, including user interactions and a
%% community-driven approach, and provide a timeline including a
%% proof-of-concept demonstration of the key software components. The
%% proposal must include a list of tangible metrics, with end user
%% involvement, to be used to measure the success of the software
%% element developed, especially the quantitative and qualitative
%% definition of a "working prototype" against which that milestone
%% will be judged, and the steps necessary to take the software element
%% from prototype to dissemination into the community as reusable
%% software resources.

The software can be viewed as six separate components: three
application implementations, the provenance library, a set of
language bindings, and a set of database adapters. The latter three
together comprise the \emph{Core Provenance Library} (CPL). We expect to issue
at least three CPL releases and two releases of each application.

Our development strategy is to construct a vertical slice of the project
as quickly as possible to put something in end-user hands quickly
to obtain user feedback, which we will incorporate into later releases.
We will begin by spending the first year addressing everything
needed to enable provenance in Gaggle and Gaggle applications.
Our rationale is that Gaggle is already widely-used, and we can gain
significant leverage and experience early with it.
We have users ready to use the new features and developers ready to
do the work.
In addition, capturing provenance for interactive use is one area in
which there is little current work.
Finally, the system is entirely in Java, which means we can focus on
only one language and one database (MySQL) for the first year.
Thus, we will direct the entire team towards this goal.
We can achieve this early milestone by leveraging our API experience in
PASS as well as significant portions
of the PASS infrastructure for provenance recording and storage.
Thus, at the end of year one, we expect to have the first release of both
the CPL and provenance-enabled Gaggle (PE-Gaggle).

Once we have deployed the first release of PE-Gaggle,
we will simultaneously solicit user and developer feedback,
monitor the growth of provenance stores,
and begin work on other CPL components.
As the Data Ingest Project and Metabolic Network Inference project
require bindings in Perl and Python, we will begin the next development
phase with bindings for those languages.
During year two, we will issue minor CPL releases that introduce
additional language support.
In parallel, we will begin to address the challenges
in formalizing and enabling provenance collection for different uses.
The ISB team will begin work on provenance integration for
Data Ingest Pipeline and Metabolic Network Inference.

By the middle to end of year two, we expect to have
significant user feedback from Gaggle users,
which will lead to the detailed design of the second version
of the CPL.
We will first address API changes to minimize the impact
to the other two applications.

The next major release will include the revised API, the full set of
language bindings, any database adapters that have been requested,
and support for all the features discussed in this proposal.
By the end of year 3, we will have the second CPL release and first
release of all applications and potentially a second release of Gaggle.

In the year four, we will solicit feedback from all application users
and design and deploy the third major version of the CPL, as well as second
(and possibly third) release of applications.
By the end of this proposal, we expect the CPL to be stable and ready for
adoption into a broad range of tools from different fields.

\section{Education and Outreach}

%% Provide an explicit outreach and education plan to allow additional
%% end user groups to take advantage of the proposed work

Harvard's emergent program in Applied Computational Science
(IACS)~\cite{iacs} and
Seltzer's own graduate courses provide the foundations of our
educational plan.
The IACS program is a master's degree program for both terminal
master's students as well as Ph.D students in domain sciences whose
work requires a significant computational component.
The program requires two core courses in computer science,
one in software engineering and one in Computing Foundations for
Computational Science, as well as multiple electives..
As part of the software engineering course, students must undertake
significant projects, and we will actively recruit students to use our
provenance library in their work.

The program also includes several computer science electives and we
expect that Seltzer's undergraduate information management course
will be a popular elective among students (she regularly has graduate
students from domain sciences audit the course).
This course also has a final project component where students are asked
to identify a real world data management problem and design and
implement a solution to the problem.
Traditionally, a few of the projects from this course become long term
useful artifacts.
In the last offering, one student developed a scheduling tool for use
by the atheletic department that was deployed this past spring.
Another student developed a new social networking paradigm that he
is attempting to commercialize,
and a third student developed a room scheduling tool that we are
hoping to use within SEAS.
When the provenance library is ready for use, we will encourage
students in this course to undertake projects 
in computational science using the provenance library.

\textcolor{magenta}{These provenance tools will also be integrated
into the Harvard Forest Summer Research Program in Ecology (an NSF REU
site since 1993, currently supported by DBI 10-03938 ``Ecological
data-model fusion and environmental forecasting for the 21st
century). For the last four years, two undergraduates have worked with
co-PI Lerner and Senior Investigator Boose on developing
RDataTracker. In 2013 and 2014, other students in this REU site
(20-30/year) use R for data analysis and we have been working with
these students to integrate RDataTracker and provenance tools into
their analysis. In August 2014, we will be submitting a renewal
proposal for our REU program. The theme for the next five years will
be collection, use, and analysis of environmental ``big data'', and
will integrate provenance tools throughout all undergraduate summer
research projects.}

\section{Results from Prior NSF Support}

%% If any PI or co-PI identified on the project has received NSF
%% funding in the past five years, information on the award(s) is
%% required. Each PI and co-PI who has received more than one award
%% (excluding amendments) must report on the award most closely related
%% to the proposal. The following information must be provided:
%% 
%% (a)	the NSF award number, amount and period of support;
%% 
%% (b)	the title of the project;
%% 
%% (c)  summary of the results of the completed work, including
%% accomplishments related to the Broader Impact activities supported
%% by the award and, for a research project, any contribution to the
%% development of human resources in science and engineering;
%% 
%% (d)	publications resulting from the NSF award;
%% 
%% (e)	evidence of research products and their availability, including, but not limited to: data, publications, samples, physical collections, software, and models, as described in any Data Management Plan; and
%% 
%% (f)	if the proposal is for renewed support, a description of the relation
%% of the completed work to the proposed work.
%% 
\subsection{PQL: A Path Query Language}

Margo Seltzer was PI on a 1--year NSF SGER grant number 0849392, NSF 08-1,
``PQL: A Path Query Language'' with a budget of \$130,000.
This grant funded preliminary work on the design and implementation of
a new query language, particularly amenable to searching graphical
structures, such as those produced by recording provenance or
lineage~\cite{pql-ipaw}.
We now have a formal semantics of the language and a prototype
implementation.
Since completion of the funded work, we have designed an update
language to augment the query language.

\subsection{Support for Atomic Sequences of File System Operations}

Margo Seltzer was Co-PI on CSR-PDOS NSF grant number 0614784 with Erez
Zadok of SUNY Stonybrook, ``CSR---PDOS: Support for Atomic Sequences
of File System Operations'' (total budget \$685,837; Harvard allocation
\$202,947).  The joint team  explored the use of atomic file system
operations to enhance the functionality and ease of development of
sophisticated services.  The provenance-aware storage system, built by
Seltzer and her students~\cite{pass-usenix06}, is a sample of such a
service.  The Harvard team has developed a collection of
provenance-aware systems including a versioning file system, a
network-attached file system, a workflow engine, an interpreter and a
browser~\cite{pass-usenix09,causalversioning-fast09,cloud-tapp09,
cloud-ladis09}.

\subsection{Harvard Forest Long Term Ecological Research}

Aaron Ellison was co-PI and Emery Boose is Senior Investigator on DEB
0620443 (`` LTER IV: Integrated studies of the drivers, dynamics, and
consequences of landscape change in New England'', \$4.92M, 6 years,
ending October 2012). Begun in 1988, the Harvard Forest LTER (HFR) is
an ongoing integrated research and educational program that examines
responses of forest dynamics to natural and human disturbances and
environmental changes over broad spatial and temporal
scales. HFR’s >60 scientists from seven institutions
investigate past, present, and future ecological patterns and
processes in New England. HFR’s observations and experiments
test fundamental ecological hypotheses; long-term studies continually
illustrate that hypotheses derived from short-term studies,
experience, or intuition are often rejected as unanticipated factors,
events and processes alter trajectories of ecological dynamics.

Central HFR findings HFR (see also http://harvardforest.fas.harvard.edu/research/LTER) include:
\begin{itemize} 
	\item Historical legacies of land use and biotic conditions 
          interacting with long-term environmental change, and natural
          and human-induced disturbances condition ecological patterns
          and processes (e.g., Foster and Aber 2004, Thompson et
          al. 2011);

	\item Climatic change and disturbance together can trigger
          abrupt ecological shifts by causing the loss of foundation
          species that control biotic and environmental conditions and
          ecosystem processes (e.g., Albani et al. 2010, Ellison et
          al. 2010, Orwig et al. 2013);

	\item Ecosystem trajectories have large inter-annual 
          variability (e.g., Urbanski et al. 2007);
	\item Strong biogeochemical resiliency to disturbance
          maintains ecosystem functions despite disturbance-induced
          changes in system structure (e.g. Ollinger et al. 2008,
          Melillo et al. 2010, Finzi et al. 2011);

	\item Effective ecological interpretation and management
          depend on integrated study of human/natural systems through
          retrospective study, decadal measurements, experiments, and
          modeling (Foster and Aber 2004)

	\item Scientists must engage early with decision-makers to
          span the science and policy boundary (Foster et al. 2010,
          Lambert 2010).  HFR research is documented in >600
          peer-reviewed publications and a synthesis volume (Foster \&
          Aber 2004). Application of these results and insights has
          directed development of state, regional, and national
          policies for forest conservation and management. HFR also is
          the NEON Domain 1 core site; plays a major role in LTER
          leadership, strategic planning, and network-wide studies;
          and has been involved with two ULTRA planning grants.

\end{itemize}

HFR research is documented in >600 peer-reviewed publications and a
synthesis volume (Foster \& Aber 2004). Application of these results
and insights has directed development of state, regional, and national
policies for forest conservation and management. HFR also is the NEON
Domain 1 core site; plays a major role in LTER leadership, strategic
planning, and network-wide studies; and has been involved with two
ULTRA planning grants.

\todo{citations are entered as comments in the tex file}

%%Literature cited:
%%Albani, M., P. R. Moorcroft, A. M. Ellison, D. A. Orwig, and D. R. Foster. 2010. Predicting the impact of hemlock woolly adelgid on carbon dynamics of Eastern U.S. forests. Canadian Journal of Forest Research 40:119-133.
%%Ellison, A. M., A. Barker Plotkin, D. R. Foster and D. A. Orwig. 2010. Experimentally testing the role of foundation species in forests: the Harvard Forest Hemlock Removal Experiment. Methods in Ecology & Evolution 1:168-179.
%%Finzi, A.C., A.T. Austin, E.E. Cleland, S.D. Frey, B.Z. Houlton and M.D. Wallenstein. 2011. Responses and feedbacks of coupled biogeochemical cycles to climate change: examples from terrestrial ecosystems. Frontiers in Ecology & Evolution 9:61-67.
%%Foster, D. R., and J. D. Aber. 2004. Forest in time: the environmental consequences of 1,000 years of change in New England. Yale University Press, New Haven, CT.
%%Lambert, K.F. (Project Consultant). 2010. LTER Strategic Communication Plan: Bridging to Broader Audiences. Long-Term Ecological Research Program Network Office. University of New Mexico, Albuquerque, NM.
%%Melillo, J. M., S. Butler, J. Johnson, J. Mohan, P. A. Steudler, H. Lux, E. Burrows, F. P. Bowles, R. Smith, L. Scott, C. Vario, T. Hill, A. J. Burton, Y. Zhou and J. Tang. 2011. Soil-warming carbon–nitrogen interactions and forest carbon budgets. Proceedings of the National Academy of Sciences of the United States of America 108:9508-9512.
%%Ollinger, S. V., A. D. Richardson, M. E. Martin, D. Y. Hollinger, S. F. Frolking, P. B. Reich, L. C. Plourde, G. G. Katul, J. W. Munger, R. Oren, M.-L. Smith, U. Paw, P. V. Bolstad, B. D. Cook, M. C. Day, T. A. Martin, R. K. Monson, and H. P. Schmid. 2008. Canopy nitrogen, carbon assimilation and albedo in temperate and boreal forests: functional relations and potential climate feedbacks. Proceedings of the National Academy of Sciences 105:19335-19340.
%%Orwig, D.A., A. A. Barker Plotkin, E. A. Davidson, H. Lux, K. E. Savage, and A. M. Ellison. 2013. Foundation species loss affects vegetation structure more than ecosystem function in a northeastern USA forest. PeerJ 1: e41.
%%Thompson, J. R., D. R. Foster, R. Scheller, and D. B. Kittredge. 2011. The influence of land use and climate change on forest biomass and composition in Massachusetts, USA. Ecological Applications 21(7):2425-2444.


\newpage \bibliographystyle{abbrv} \bibliography{prop}

\end{document}
